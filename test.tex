\documentclass{article}
\usepackage{graphicx} % Required for inserting images
\usepackage{amsmath}
\usepackage[english, russian]{babel} 

\begin{document}

\begin{center}
\Large
\textbf{Простейшие задачи}
\end{center}

\item{Отсюда видно, что в точке приложения сосреточенной силы первые производные претерпевают разрыв и дифферециальное уравнение теряет смысл. В этой точке должны выполняться два условия сопряжения}

\item{
\left.
\begin{array}{ccc}
     x & = &\[u(x_0 + 0, f) == u(x_0 - 0, f),\] \\
     y & = &\[u_x(x_0 + 0, f) - u_x (x_0 - 0, f) == -\frac{1}{T_a}f_0(f),\] \\
\end{array}
\right\}

\item{первое из которых выражает непрерывность струны, второе определяет велечину излома струны в точке $x_0$, зависящую от $f_0(t)$ и натяжению $T_0$.}

\textbf{2. Уравнение продольных колебаний стержней и струн.} Уравнения продольных колебаний для струны, стержня и пружины записываются одинаково. Рассмотрим стержень, расположенный на отрезке $(0, l)$ оси x. Процесс продольных колебаний может быть описан одной функцией $u(x,t)$, представляющей в момент $t$ смещение точки, имевшей в положении равновесия абциссу $(x^1)$. При продольных колебаниях это смещение происходит вдоль стержня. При выводе уравнения будем предополгать, что натяжения, возникающие в процессе колебания, следуют закону Гука.

\textbf{} Подсчитаем относительное удлинение элемента $(x, x + \Delta x)$ в момент $t$. Координаты концов этого элемента в момент $t$ имеют значения \[x + u (x, t), x + \Delta x + u (x + \Delta x, t)\] а относительное удлинение равно \[\frac{[\Delta x + u (x + \Delta x, t) - u (x, t] - \Delta x}{\Delta x} = u_x (x + 0 \Delta x, t)\] 

\[(0 \leq 0 \leq 1)\]

\textbf{} Переходя к пределу при $\Delta x \rightarrow 0$, получим, что относительное удлинение в точке x определяется функцией $u_x (x, t)$. В силу закона Гука натяжение $T (x, t)$ равно

\[T (x, t) = k(x) u_x (x, t),\] где $k(x)$ - модуль Юнга в точке $x(k(x) > 0).$

\textbf{} Пользуясь теоремой об изменении количества движения, получаем интегральное уравнение колебаний

\[\int_{x_1}^{x_2}[u_t (\xi, t_2) - u_t(\xi, t_1)]p (\xi)d\xi =\]
\[=\int_{i_1}^{t_2}[k(x_2)u_x(x_2, \tau) - k(x_1)u_x(x_1)u_x(x_1, \tau)]d\tau + \int_{x_1}^{x_2}\int_{t_1}^{t_2}F(\xi, \tau)d\xi d\tau\] где $F(x, t)$ - плотность внешней силы, рассчитанная на единицу длины.

\textbf{} Предположим существование и непрерывность вторых производных функций $u(x, t)$. Применяя теорему о среднем и совершая предельный переход) при $\Delta x = x_2 - x_1 \rightarrow 0$ и $\Delta t = t_2 - t_1 \rightarrow 0,$ приходим к дифференциальному уравнению продольных колебаний стержня)

\[[k(x)u_x]_x = p u_t_t - F(x, t)\] Если стержень однороден $(k(x) = const),$ то это уравнение записывают следующим образом:

\[u_t_t = a^2 u_x_x + f(x, t) (a = \sqrt{\frac{k}{p}}),\] где \[f(x,t) = \frac{F(x, t)}{p}\] есть плотность силы, отнесенная к единице массы.

\textbf{3. Энергия колебаний струны.} Найдём выражение для энергии поперечных колебаний струны $E = K + U,$ где K - кинитеческая и U - потенциальная энергия. Элемент струны $dx$, движущийся со скоростью $v = u_t$, обладает кинитической энергией\[\frac{1}{2}m v^2 = \frac{1}{2}p(x)dx(u_t)^2 \, \qquad (m=pdx)\]

\usepackage{amsmath}

\begin{equation}
    f(x) = 
    \begin{cases}
        0, & \text{если } x < 0 \\
        1, & \text{если } x \geq 0
    \end{cases}
\end{equation}
 
\end{document}
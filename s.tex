\documentclass{article}
\usepackage[russian]{babel}
\begin{document}
\begin{center}
\footnotesize{}
ҚЛАССИФИКАЦИЯ ДИФФЕРЕНЦИАЛЬНЫХ УРАВНЕНИЙ
\end{center}
Таким образом, если уравнение (12) в некоторой точке М
принадлежит к определенному типу, то его можно привести к
соответствующей канонической форме в этой точке.

Рассмотрим подробнее вопрос о том, можно ли привести
уравнение к канонической форме в некоторой окрестности точки
М, если во всех точках этой окрестности уравнение принадлежит к одному и тому же типу.

Для приведения уравнения в некоторой области к каноническому виду нам пришлось бы функции $\xi(x_1, x_2, ..., x_n$) (i = 1, 2, ..., n) подчинить дифференциальным соотношениям $ \overline{a}_k_l = 0, для k \neq 1.$ Число этих условий, равное n(n-1)/2, превосходит n - число определяемых функций $\xi$ при n > 3.  Для n = 3  недиагональные элементы матрицы ($\overline{a}_k_l),$ вообще говоря можно было бы обратить в нули, но при этом диагональные элементы могут оказаться различными.

Следовательно, при n $\geq$ 3 уравнение нельзя привести к каноническому виду в окрестности точки M. При n = 22 можно обратить в нуль единственный недиагональный коэффициент и удовлетыорить условию равенства двух диагональных коэффициентов, что и было сделано в п. 1.

Если коэффициенты уравнения (12) постоянны, то, приводя (1) к канонической форме в одной точке M, мы получим уравнение, привеленное к канонической форме во всей области определения уравнения.

\textbf{3. Канонические формы линейных уравнений с постоянными коэффициентами.} В случае двух независимых переменных линейное уравнение 2-го порядка с постоянными коэффициентами имеет вид 

\[a_1_1u_x_x+2a_1_2u_y_y+a_2_2u_y_y+b_1u_x+b_2_u_y+cu+f(x, y)=0\]
Ему соответствует характеристическое уравнение с постоянными коэффициентами. Поэтому \item{
\left.
\begin{array}{ccc}
     y & = &\[\frac{a_1_2+\sqrt{a^2_1_2-a_1_1a_2_2}}{a_1_1} x+C_1,\] \\
     y & = &\[\frac{a_{12}-\sqrt{a^2_{12} - a_1_1a_2_2}}{a_1_1}x+C_2.\] (14) \\
\end{array} 
\right\} 

С помощью соответствующего преобразования переменных уравнения (14) приводится к одной из простейших форм:
\item{
\begin{flushleft}
$u_{\xi\xi}+u_{\eta\eta}+b_1u_\eta+b_2u_\eta+cu+f=0$ (эллиптический тип), (15)

\left.
\begin{array}{ccc}
    && $u_{\xi\ta}+b_1u_\xi+b_2u_\eta+cu+f=0$ \\
    $или$ \\
    && $u_{\xi\xi}-u_{\eta\eta}+b_1u_\xi+b_2u_eta+cu+f=0$ \\
\end{array}
\right\}
\newline
$u_{\xi\xi}+b_1u_\xi+b_2u_\eta+cu+f=0$ (параболический тип). (17)
\item
\item
\end{flushleft}
}





Для дальнейшего упрощения введем вместо u новю функцию $\upsilon$:
\[u=e^{\gamma\xi+\mu\eta}\cdot\upsilon,\]
где $\gamma$ и $\mu$ - неопределенные пока постоянные. Тогда
\[u_\xi=e^{\gamma\xi+\mu\eta}(\upsilon_\xi+\gamma\upsilon),\]
\[u_\eta=e^{\gamma\xi+\mu\eta}(\upsilon_\eta+\mu\upsilon),\]
\[u_\xi_\xi=e^{\gamma\xi+\mu\eta}(\upsilon_\xi_\xi+2\gamma\upsilon_\xi+\gamma^2\upsilon),\]
\[u_\xi_\eta=e^{\gamma\xi+\mu\eta}(\upsilon_\xi_\eta+\gamma\upsilon_\eta+\mu\upsilon_\xi+\gamma\mu\upsilon),\]
\[u_\xi=e^{\gamma\xi+\mu\eta}(\upsilon_\eta_\eta+2\mu\upsilon_\eta+\mu^2\upsilon).\]
Подставляя выражения для производных в уравнение (15) и сокращая затем на $e^{\gamma\xi+\mu\eta}$, получим:
\[\upsilon_{\xi\xi}+\upsilon_\eta_\eta+(b_1+2\gamma)\upsilon_\xi+(b_2+2\mu)u_\eta+\]
\[+(\gamma^2+\mu^2+b_1\gamma+b_2\mu+c)\upsilon+f_1=0.\]
Параметры $\gamma$ и $\mu$ выбираем так, чтобы два коэффициента, например, при первых производных, обратились в нуль $(\gamma=-b_1/2$; $\mu=-b_2/2)$. В результате получим:
\[\upsilon_\xi_\xi+\upsilon_{\eta\eta}+\gamma\upsilon+f_1=0,\]
где $\gamma$ - постоянная, выражающая через c, $b_1$, и $b_2, f_1=fe^{-(\gamma\xi+\mu\eta)}.$ Производя аналогичные операции и для случаев (16) и (17), приходим к следующим каноническим формам для уравнений с постоянными коэфициентами:

$\upsilon_{\xi\xi}+\upsilon_{\eta\eta}+\gamma\upsilon+f_1=0$ (эллиптический тип),
\[\upsilon_{\xi\eta}+\gamma\upsilon+f_1=0\]
\[\upsilon_\xi\xi-\upsilon_{\eta\eta}+\gamma\upsilon+f_1=0\]
\[\upsilon_{\xi\xi}+b_2\upsilon_\eta+f_1=0 (параболический тип).\]
Как было отмечено в п. 2, уравнение с постоянными коэффициентами в случае нескольких независимых переменных
\[n=\sum\limits_{i=1}^{n}\sum\limits_{i=1}^{n}a_i_j\upsilon_{x_ix_j}+\sum\limits_{i=1}^{n}b_i\upsilon_{x_i}+cu+f=0\]
при помощи линейного преобразования переменных приводится к каноническому виду одновремменно для всех точек области его опрелеления. Вводя вместо u новую функцию $\upsilon$
\[u=\upsilon e^{\gamma_1x_1+\gamma_2x_2+ ... +\gamma_nx_n}\]

\end{document}